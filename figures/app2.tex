\begin{figure}[bp]
   \begin{tikzpicture}
      \begin{groupplot}
      [group style={columns=2, horizontal sep=2cm},
      xmin=0, xmax=0.6, ymin=0, ymax=0.6]
         \nextgroupplot[xlabel=$\Delta^{(I)}_\eta$, ylabel=$\Delta^{(II)}_\eta$]
         \addplot [black!10, no markers] coordinates {(0,0) (0.6,0.6)};
         \addplot [only marks] table [x = d1, y = d2] {figures/app2.dat};
         \node[black!80, below left] at (axis cs:0.1,0.55){\textbf{A}};
         \nextgroupplot[xlabel=$\Delta^{(III in)}_\eta$, ylabel=$\Delta^{(III out)}_\eta$]
         \addplot [black!10, no markers] coordinates {(0,0) (0.6,0.6)};
         \addplot [only marks] table [x = d3i, y = d3o] {figures/app2.dat};
         \node[black!80, below left] at (axis cs:0.1,0.55){\textbf{B}};
      \end{groupplot}
   \end{tikzpicture}

   \caption{Results of the null model analysis of 59 plant-pollination networks.
   \textbf{A}. There is a consistent tendency for (i) both models I and II to
   estimate less nestedness than in the empirical network, although null model
   II yields more accurate estimates. \textbf{B}. Models III in and III out also
   estimate less nestedness than the empirical network, but neither has a
   systematic bias. For each null model $i$, the difference $\Delta^{(i)}_\eta$
   in nestedness $\eta$ is expressed as $\Delta^{(i)}_\eta =
   \eta-\mathcal{N}^{(i)}(\eta)$, where $\mathcal{N}^{(i)}(\eta)$ is the
   nestedness of null model $i$.}

   \label{f:app2}
\end{figure}
